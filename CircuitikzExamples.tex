\documentclass[12pt]{exam}

\usepackage[margin=1in]{geometry}

\newcommand{\class}{PHYS 202}
\newcommand{\term}{Spring 2013}
\newcommand{\examnum}{CO9 (v1)}
\newcommand{\examdate}{\today}
\newcommand{\timelimit}{50 Minutes}


% Engine-specific settings
% Detect pdftex/xetex/luatex, and load appropriate font packages.
% This is inspired by the approach in the iftex package.
% pdftex:
\ifx\pdfmatch\undefined
\else
    \usepackage[T1]{fontenc}
    \usepackage[utf8]{inputenc}
\fi
% xetex:
\ifx\XeTeXinterchartoks\undefined
\else
    \usepackage{fontspec}
    \defaultfontfeatures{Ligatures=TeX}
\fi
% luatex:
\ifx\directlua\undefined
\else
    \usepackage{fontspec}
\fi
% End engine-specific settings

\usepackage{amsmath,amssymb}
\usepackage{fullpage}
\usepackage{graphicx}
\usepackage[svgnames]{xcolor}
\usepackage{url}
\urlstyle{same}

%\usepackage{pythontex}
% \restartpythontexsession{\thesection}


\usepackage[framemethod=TikZ]{mdframed}

\newcommand{\pytex}{Python\TeX}
\renewcommand*{\thefootnote}{\fnsymbol{footnote}}

\usepackage{circuitikz}
\usepackage{siunitx}
\usepackage{pgfplots}
\usepackage{nopageno}



\begin{document}


\begin{figure}[!h]
\begin{center}\begin{circuitikz}\draw
  (2,3.5) to[short] ++(0.5,0)
  to[resistor] ++(0,2)
  (2.5,5.5) to[short] ++(2,0)
  (4.5,5.5) to[resistor] ++(0,-2)
  to[resistor] ++(0,-2)
  (4.5,1.5) to[short] ++(-2,0)
  (2.5,1.5) to[resistor] ++(0,2)
  (4.5,3.5) to[resistor] ++(2.5,0)
  (7,3.5) node[circ]{}
  (-1.5,3.5) node[circ,label=left:A]{}
  (-1.5,3.5) to[resistor] ++(1.5,0)
  to[resistor] ++(2,0)
  (7,3.5) to[resistor] ++(2.5,0)
;\end{circuitikz}\end{center}
\end{figure} `

\begin{figure}[!h]
\begin{center}\begin{circuitikz}\draw
  (0,3) to[resistor] ++(3,0)
  to[resistor] ++(0,-3)
  to[resistor] ++(-3,0)
  to[battery, invert] ++(0,3)
;\end{circuitikz}\end{center}
\end{figure} 


\begin{figure}[!h]
\begin{center}\begin{circuitikz}\draw
  (3,3) to[resistor] ++(0,-3)
  (9,0) to[short] ++(-9,0)
  to[battery, invert] ++(0,3)
  (6,3) to[resistor] ++(0,-3)
  (0,3) to[short] ++(9,0)
  to[resistor] ++(0,-3)
;\end{circuitikz}\end{center}
\end{figure} 

% Take above figure and put in all coordinates explicitly:
\begin{figure}[!h]
\begin{center}\begin{circuitikz}\draw
  (3,3) to[resistor=$R_1$] (3,0)
  (9,0) to[short] (0,0)
  (0,0) to[battery, invert] (0,3)
  (6,3) to[resistor=$R_2$] (6,0)
  (0,3) to[short] (9,3)
  (9,3) to[resistor=$R_3$] (9,0)
;\end{circuitikz}\end{center}
\end{figure} 

% Draw outer loop then inner resistors:
\begin{figure}[!h]
\begin{center}\begin{circuitikz}\draw
  (0,0) to[battery=$V_b$, invert] (0,3)
  to[short] (9,3)
  to[resistor=$R_3$] (9,0)
  to[short] (0,0)
  
  (3,3) to[resistor=$R_1$] (3,0)
  (6,3) to[resistor=$R_2$] (6,0)
  
;\end{circuitikz}\end{center}
\end{figure} 

\begin{figure}[!h]
\begin{center}\begin{circuitikz}\draw
	(0,0) node[circ,label=left:B]{}
	to[resistor=$R_1$] (3,0)
	to[resistor=$R_1$] (4.5,0)
	to[resistor=$R_1$] (6,0)
	to[resistor=$R_1$] (7.5,0)
	to[resistor=$R_1$] (9,0)
	to[resistor, l_=$R_2$] (9,3)
	to[resistor=$R_1$] (7.5,3)
	to[resistor=$R_1$] (6,3)
	to[resistor=$R_1$] (4.5,3)
	to[resistor=$R_1$] (3,3)
	to[resistor=$R_1$] (0,3)
	(0,3) node[circ,label=left:A]{}
	(6,3) to[resistor=$R_3$] ++(0,-3)
	(3,3) to[resistor=$R_3$] ++(0,-3)
;\end{circuitikz}\end{center}
\end{figure}


\begin{figure}[!h]
\begin{center}\begin{circuitikz}\draw
	(0,0) to[battery, invert,l=$V_a$] (0,3)
	to[resistor=$R_1$] (3,3)
	to[resistor=$R_2$] (6,3)
	to[short] (6,0)
	to[resistor=$R_1$] (3,0)
	to[short] (0,0)
	(3,0) to[battery, invert,l=$V_b$] (3,3)


;\end{circuitikz}\end{center}
\end{figure}

This is a test.

\begin{figure}[!h]
\begin{center}\begin{circuitikz}\draw
	(0,0) to[battery, invert,l=$V$] (0,3)
	to[resistor=$R_A$] (3,3)
	to[short] (4.5,3)
	to[resistor=$R_C$] (7.5,3)
	to[short] (7.5,0)
	to[short] (3,0)
	to[short] (0,0)
	(3,1.5) to[short] (3,4.5)
	to[switch](4.5,4.5)
	to[resistor=$R_B$] (7.5,4.5)
	to[short] (7.5,1.5)
	to[resistor,l_=$R_D$] (4.5,1.5)
	to[switch,invert] (3,1.5)

;\end{circuitikz}\end{center}
\end{figure}


\begin{figure}[!h]
\begin{center}\begin{circuitikz}\draw
	(0,0) to[short] (0,1.5)
	to[battery, invert,l=$V_1$] (0,3)
	to[short] (3,3)
	to[resistor=$R_3$] (6,3)
	to[battery,l=$V_3$] (6,1.5)
	to[short] (6,0)
	to[short] (3,0)
	to[resistor=$R_1$] (0,0)
	(3,0) to[resistor=$R_2$] (3,1.5)
	to[battery, invert,l=$V_2$] (3,3)
;\end{circuitikz}\end{center}
\end{figure}



\end{document}