\documentclass[12pt]{exam}

\usepackage[margin=1in]{geometry}

\newcommand{\class}{PHYS 202}
\newcommand{\term}{Spring 2013}
\newcommand{\examnum}{CO13 (v1)}
\newcommand{\examdate}{\today}
\newcommand{\timelimit}{50 Minutes}


% Engine-specific settings
% Detect pdftex/xetex/luatex, and load appropriate font packages.
% This is inspired by the approach in the iftex package.
% pdftex:
\ifx\pdfmatch\undefined
\else
    \usepackage[T1]{fontenc}
    \usepackage[utf8]{inputenc}
\fi
% xetex:
\ifx\XeTeXinterchartoks\undefined
\else
    \usepackage{fontspec}
    \defaultfontfeatures{Ligatures=TeX}
\fi
% luatex:
\ifx\directlua\undefined
\else
    \usepackage{fontspec}
\fi
% End engine-specific settings

\usepackage{amsmath,amssymb}
\usepackage{fullpage}
\usepackage{graphicx}
\usepackage[svgnames]{xcolor}
\usepackage{url}
\urlstyle{same}

\usepackage{pythontex}
% \restartpythontexsession{\thesection}


\usepackage[framemethod=TikZ]{mdframed}

\newcommand{\pytex}{Python\TeX}
\renewcommand*{\thefootnote}{\fnsymbol{footnote}}

\usepackage{circuitikz}
\usepackage{siunitx}
\usepackage{pgfplots}
\usepackage{nopageno}



\begin{document}

\noindent
\begin{tabular}{l l r}
\textbf{\class} - \textbf{\examnum}  & \textbf{Name} \makebox[2.25in]{\hrulefill} & \textbf{i.d.} \makebox[1.0in]{\hrulefill} \\
%\textbf{\term} &&\\
%\textbf{\examnum} &&\\
%\textbf{\examdate} &&\\
%\textbf{Time Limit: \timelimit} & Teaching Assistant & \makebox[2in]{\hrulefill}
\end{tabular}\\
\rule[1ex]{\textwidth}{2.21pt}

\pagestyle{head}
\firstpageheader{}{}{}
\runningheader{\class}{\examnum\ }{Page \thepage\ of \numpages}
\runningheadrule


\begin{questions}
\question


\begin{pycode}
from pylab import *
import random

from randassign import RandAssign
ra = RandAssign()

def make_problem():
	# Random values of circuit components
	def rand_battery():
		return random.choice([1.5, 3, 4.5, 6, 9, 12, 15, 18])
	def rand_R1():
		return random.choice([100, 120, 150, 180, 220])
	def rand_R2():
		return random.choice([220, 270, 330, 390, 470, 560, 680, 820, 1000])
	def rand_R3():
		return random.choice([470, 560, 680, 820, 1000])	
	def rand_choice1():
		return random.choice(["current through","potential difference across"])
	def rand_choice2():
		return random.choice(["resistor=$R_D$","short"])
	def rand_choice3():
		return random.choice(["greatest","smallest"])

		
	R1 = rand_R2()
	R2 = rand_R2()
	R3 = rand_R2()
	R4 = rand_R1()
	R5 = rand_R1()

	Vb = rand_battery()
	question_choice = random.choice(["$R_1$", "$R_1$", "$R_1$", "$R_1$"])
		
	problem_template = r'''
	\begin{{figure}}[!h]
	\begin{{center}}\begin{{circuitikz}}\draw
		(0,0) to[battery,invert,l=$V_b$] (0,3)
		to[resistor=$R_1$] (3,3)
		to[resistor=$R_2$] (3,0)
		to[resistor=$R_4$] (0,0)
	(0,3)	to[resistor=$R_3$] (3,0)
	;\end{{circuitikz}}\end{{center}}
	\end{{figure}}

    \textbf{{Given the following values}}:
    $V_b=\SI{{{4:.3g}}}{{\volt}}$,
    $R_1=\SI{{{0:.4g}}}{{\ohm}}$,
    $R_2=\SI{{{1:.4g}}}{{\ohm}}$,
    $R_3=\SI{{{2:.4g}}}{{\ohm}}$, and
    $R_4=\SI{{{3:.4g}}}{{\ohm}}$, determine the power dissipated by {{{5}}}.
	'''
	
	## NO SOLUTION CODE HAS BEEN WRITTEN (YET)
	R12 = R1+R2
	R123 = 1/((1/R3)+(1/R12))
	Req = R123 + R4
	I4 = Vb/Req
	V4 = I4*R4
	V3 = Vb-V4
	V12 = V3
	I1 = V12/R12
	I2 = I1
	I3 = V3/R3
	V1 = I1*R1
	V2 = I2*R2
	W1 = I1**2*R1
	W2 = I2**2*R2
	W3 = I3**2*R3
	W4 = I4**2*R4
	problem = problem_template.format(R1,R2,R3,R4,Vb,question_choice)
	
	return(problem,V1,V2,V3,V4,I1,I2,I3,I4,W1,W2,W3,W4)
	
## Generate 4 unique problems
## The chance of two identical problems is very low in this case
## But checking for identical problems won't hurt
#problems = []
#while len(problems) < 4:
#    p,v1,v2,v3,v4,v5,i1,i2,i3,i4,i5 = make_problem()
#    if p not in problems:
#        problems.append(p)
#        t = '\\SI{{{0:.3g}}}{{\\volt}} and \\SI{{{1:.3g}}}{{\\volt}}and \\SI{{{2:.3g}}}{{\\volt}} and \\SI{{{3:.3g}}}{{\\volt}} and \\SI{{{4:.3g}}}{{\\volt}} and \\SI{{{5:.3g}}}{{\\ampere}} and \\SI{{{6:.3g}}}{{\\ampere}} and \\SI{{{7:.3g}}}{{\\ampere}} and \\SI{{{8:.3g}}}{{\\ampere}} and \\SI{{{9:.3g}}}{{\\ampere}}'
#        ra.addsoln(t.format(v1,v2,v3,v4,v5,i1,i2,i3,i4,i5))

#print('\\begin{enumerate}')
#for n, p in enumerate(problems):
#    print('\\item ' + p)
#    if n+1 < len(problems):
#        print(r'\vspace{1in}')
#print('\\end{enumerate}')

p,v1,v2,v3,v4,i1,i2,i3,i4,w1,w2,w3,w4 = make_problem()
t = '\\SI{{{0:.3g}}}{{\\volt}} and \\SI{{{1:.3g}}}{{\\volt}}and \\SI{{{2:.3g}}}{{\\volt}} and \\SI{{{3:.3g}}}{{\\volt}} and \\SI{{{4:.3g}}}{{\\ampere}} and \\SI{{{5:.3g}}}{{\\ampere}} and \\SI{{{6:.3g}}}{{\\ampere}} and \\SI{{{7:.3g}}}{{\\ampere}}, and \\SI{{{8:.3g}}}{{\\watt}}, and \\SI{{{9:.3g}}}{{\\watt}}, and \\SI{{{10:.3g}}}{{\\watt}} and \\SI{{{11:.3g}}}{{\\watt}}'
ra.addsoln(t.format(v1,v2,v3,v4,i1,i2,i3,i4,w1,w2,w3,w4))

print(p)

\end{pycode}


\end{questions}
	
\end{document}
