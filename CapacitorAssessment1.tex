\documentclass[12pt]{exam}

\usepackage[margin=1in]{geometry}

\newcommand{\class}{PHYS 202}
\newcommand{\term}{Spring 2013}
\newcommand{\examnum}{CO9 (v1)}
\newcommand{\examdate}{\today}
\newcommand{\timelimit}{50 Minutes}


% Engine-specific settings
% Detect pdftex/xetex/luatex, and load appropriate font packages.
% This is inspired by the approach in the iftex package.
% pdftex:
\ifx\pdfmatch\undefined
\else
    \usepackage[T1]{fontenc}
    \usepackage[utf8]{inputenc}
\fi
% xetex:
\ifx\XeTeXinterchartoks\undefined
\else
    \usepackage{fontspec}
    \defaultfontfeatures{Ligatures=TeX}
\fi
% luatex:
\ifx\directlua\undefined
\else
    \usepackage{fontspec}
\fi
% End engine-specific settings

\usepackage{amsmath,amssymb}
\usepackage{fullpage}
\usepackage{graphicx}
\usepackage[svgnames]{xcolor}
\usepackage{url}
\urlstyle{same}

\usepackage{pythontex}
% \restartpythontexsession{\thesection}


\usepackage[framemethod=TikZ]{mdframed}

\newcommand{\pytex}{Python\TeX}
\renewcommand*{\thefootnote}{\fnsymbol{footnote}}

\usepackage{circuitikz}
\usepackage{siunitx}
\usepackage{pgfplots}
\usepackage{nopageno}



\begin{document}

\noindent
\begin{tabular}{l l r}
\textbf{\class} - \textbf{\examnum}  & \textbf{Name} \makebox[2.25in]{\hrulefill} & \textbf{i.d.} \makebox[1.0in]{\hrulefill} \\
%\textbf{\term} &&\\
%\textbf{\examnum} &&\\
%\textbf{\examdate} &&\\
%\textbf{Time Limit: \timelimit} & Teaching Assistant & \makebox[2in]{\hrulefill}
\end{tabular}\\
\rule[1ex]{\textwidth}{2.21pt}

\pagestyle{head}
\firstpageheader{}{}{}
\runningheader{\class}{\examnum\ }{Page \thepage\ of \numpages}
\runningheadrule


\begin{questions}
\question Consider the following circuit:


\begin{figure}[!h]
\begin{center}\begin{circuitikz}\draw
  (1,4.5) to[capacitor, l_=$C_1$] ++(0,-2)
  (1,4.5) to[capacitor,l=$C_2$] ++(2.5,0)
  to[capacitor,l=$C_3$] ++(0,-2)
  (3.5,2.5) to[short] ++(-2.5,0)
  to[short] ++(-2,0)
  (1,4.5) to[short] ++(-2,0)
  (-1,4.5) to[battery,l_=$V_b$] ++(0,-2)
;\end{circuitikz}\end{center}
\end{figure}



\begin{pycode}
from pylab import *
import random

from randassign import RandAssign
ra = RandAssign()

def make_problem():
	# Random values of circuit components
	def rand_capacitor():
		return random.choice([15, 22, 33, 47, 68, 100, 150, 220, 330, 470, 680])
	def rand_battery():
		return random.choice([1.5, 3, 4.5, 6, 9, 12])
		
	C1 = rand_capacitor()*1e-6
	C2 = rand_capacitor()*1e-6
	C3 = rand_capacitor()*1e-6
	V1 = rand_battery()
	
	# Find equivalent capacitance between C2 and C3:
	C23 = 1 / ((1/C2)+(1/C3))
	C123 = C23+C1
	
	#Find charges
	Q23 = C23*V1
	Q1 = C1*V1
	Q2 = Q23
	Q3 = Q23
	
	#Find potentials:
	V2 = Q2 / C2
	V3 = Q3 / C3
	
	#Find Energy stored:
	U1 = .5*C1*V1**2
	U2 = .5*C2*V2**2
	U3 = .5*C3*V3**2
	
	problem_template = r'''
    \textbf{{Given the following values}}:
    $V_b=\SI{{{3:.3g}}}{{\volt}}$,
    $C_1=\SI{{{0:.3g}}}{{\micro\farad}}$,
    $C_2=\SI{{{1:.3g}}}{{\micro\farad}}$, and
    $C_3=\SI{{{2:.3g}}}{{\micro\farad}}$, find the charge on each capacitor, the potential difference across each capacitor, and the energy stored in each capacitor.
	'''
	
	problem = problem_template.format(C1/1e-6,C2/1e-6,C3/1e-6,V1)
	
	return(problem, C123, Q1, Q2, Q3, V1, V2, V3, U1, U2, U3)
	
## Generate 4 unique problems
## The chance of two identical problems is very low in this case
## But checking for identical problems won't hurt
#problems = []
#while len(problems) < 4:
#    p, c123, q1, q2, q3, v1, v2, v3, u1, u2, u3 = make_problem()
#    if p not in problems:
#        problems.append(p)
#        t = '\\SI{{{0:.3g}}}{{\\micro\\farad}} and \\SI{{{1:.3g}}}{{\\micro\\coulomb}}and \\SI{{{2:.3g}}}{{\\micro\\coulomb}} and \\SI{{{3:.3g}}}{{\\micro\\coulomb}} and \\SI{{{4:.3g}}}{{\\volt}} and \\SI{{{5:.3g}}}{{\\volt}} and \\SI{{{6:.3g}}}{{\\volt}} and \\SI{{{7:.3g}}}{{\\joule}} and \\SI{{{8:.3g}}}{{\\joule}} and \\SI{{{9:.3g}}}{{\\joule}}'
#        ra.addsoln(t.format(c123, q1, q2, q3, v1, v2, v3, u1, u2, u3))
#
#print('\\begin{enumerate}')
#for n, p in enumerate(problems):
#    print('\\item ' + p)
#    if n+1 < len(problems):
#        print(r'\vspace{1in}')
#print('\\end{enumerate}')

p, c123, q1, q2, q3, v1, v2, v3, u1, u2, u3 = make_problem()
t = '\\SI{{{0:.3g}}}{{\\micro\\farad}} and \\SI{{{1:.3g}}}{{\\micro\\coulomb}}and \\SI{{{2:.3g}}}{{\\micro\\coulomb}} and \\SI{{{3:.3g}}}{{\\micro\\coulomb}} and \\SI{{{4:.3g}}}{{\\volt}} and \\SI{{{5:.3g}}}{{\\volt}} and \\SI{{{6:.3g}}}{{\\volt}} and \\SI{{{7:.3g}}}{{\\joule}} and \\SI{{{8:.3g}}}{{\\joule}} and \\SI{{{9:.3g}}}{{\\joule}}'
ra.addsoln(t.format(c123, q1, q2, q3, v1, v2, v3, u1, u2, u3))

print(p)

\end{pycode}

\newpage
\question Find the equivalent capacitance for the following network of capacitors:

\begin{figure}[!h]
\begin{center}\begin{circuitikz}\draw
  (2,3.5) to[short] ++(0.5,0)
  to[capacitor, l_=$C_3$] ++(0,2)
  (2.5,5.5) to[short] ++(4,0)
  (6.5,5.5) to[capacitor, l_=$C_4$] ++(0,-2)
  to[capacitor, l_=$C_6$] ++(0,-2)
  (6.5,1.5) to[short] ++(-4,0)
  (2.5,1.5) to[capacitor, l_=$C_5$] ++(0,2)
  (6.5,3.5) to[capacitor, l_=$C_7$] ++(2.5,0)
  (9,3.5) node[circ, label=right:$b$]{}
  (-1.5,3.5) node[circ, label=left:$a$]{}
  (-1.5,3.5) to[capacitor, l_=$C_1$] ++(1.5,0)
  to[capacitor, l_=$C_2$] ++(2,0)
;\end{circuitikz}\end{center}
\end{figure} 


\begin{pycode}
from pylab import *
import random

from randassign import RandAssign
ra = RandAssign()

def make_problem():
	# Random values of circuit components
	def rand_capacitor():
		return random.choice([15, 22, 33, 47, 68, 100, 150, 220, 330, 470, 680])
	C1 = rand_capacitor()*1e-6
	C2 = rand_capacitor()*1e-6
	C3 = rand_capacitor()*1e-6
	C4 = rand_capacitor()*1e-6
	C5 = rand_capacitor()*1e-6
	C6 = rand_capacitor()*1e-6
	C7 = rand_capacitor()*1e-6
	# Find equivalent capacitance between C3 and C4:
	C34 = 1 / ((1/C3)+(1/C4))
	# Find equivalent capacitance between C5 and C6:
	C56 = 1 / ((1/C5)+(1/C6))
	
	C3456 = C34+C56
	Ceq = 1 / ((1/C1)+(1/C2)+(1/C3456)+(1/C7))
	
	problem_template = r'''
    \textbf{{Given the following values}}:
    $C_1=\SI{{{0:.3g}}}{{\micro\farad}}$,
    $C_2=\SI{{{1:.3g}}}{{\micro\farad}}$,
    $C_3=\SI{{{2:.3g}}}{{\micro\farad}}$,
    $C_4=\SI{{{3:.3g}}}{{\micro\farad}}$,
    $C_5=\SI{{{4:.3g}}}{{\micro\farad}}$, and
    $C_6=\SI{{{5:.3g}}}{{\micro\farad}}$,
    $C_7=\SI{{{6:.3g}}}{{\micro\farad}}$.
    '''
	problem = problem_template.format(C1/1e-6,C2/1e-6,C3/1e-6,C4/1e-6,C5/1e-6,C6/1e-6,C7/1e-6)
	
	return(problem, Ceq)

p, ceq = make_problem()
t = '\\SI{{{0:.3g}}}{{\\micro\\farad}}'
ra.addsoln(t.format(ceq))

print(p)
	

	
\end{pycode}
\end{questions}
	
\end{document}
