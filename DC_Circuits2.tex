\documentclass[12pt]{exam}

\usepackage[margin=1in]{geometry}

\newcommand{\class}{PHYS 202}
\newcommand{\term}{Spring 2013}
\newcommand{\examnum}{CO10,CO11,CO12 (v1)}
\newcommand{\examdate}{\today}
\newcommand{\timelimit}{50 Minutes}


% Engine-specific settings
% Detect pdftex/xetex/luatex, and load appropriate font packages.
% This is inspired by the approach in the iftex package.
% pdftex:
\ifx\pdfmatch\undefined
\else
    \usepackage[T1]{fontenc}
    \usepackage[utf8]{inputenc}
\fi
% xetex:
\ifx\XeTeXinterchartoks\undefined
\else
    \usepackage{fontspec}
    \defaultfontfeatures{Ligatures=TeX}
\fi
% luatex:
\ifx\directlua\undefined
\else
    \usepackage{fontspec}
\fi
% End engine-specific settings

\usepackage{amsmath,amssymb}
\usepackage{fullpage}
\usepackage{graphicx}
\usepackage[svgnames]{xcolor}
\usepackage{url}
\urlstyle{same}

\usepackage{pythontex}
% \restartpythontexsession{\thesection}


\usepackage[framemethod=TikZ]{mdframed}

\newcommand{\pytex}{Python\TeX}
\renewcommand*{\thefootnote}{\fnsymbol{footnote}}

\usepackage{circuitikz}
\usepackage{siunitx}
\usepackage{pgfplots}
\usepackage{nopageno}



\begin{document}

\noindent
\begin{tabular}{l l r}
\textbf{\class} - \textbf{\examnum}  & \textbf{Name} \makebox[2.25in]{\hrulefill} & {\hrulefill} \\
%\textbf{\term} &&\\
%\textbf{\examnum} &&\\
%\textbf{\examdate} &&\\
%\textbf{Time Limit: \timelimit} & Teaching Assistant & \makebox[2in]{\hrulefill}
\end{tabular}\\
\rule[1ex]{\textwidth}{2.21pt}

\pagestyle{head}
\firstpageheader{}{}{}
\runningheader{\class}{\examnum\ }{Page \thepage\ of \numpages}
\runningheadrule


\begin{questions}
\question Consider the following circuit:


\begin{figure}[!h]
\begin{center}\begin{circuitikz}\draw
	(0,0) to[battery, invert,l=$V$] (0,3)
	to[resistor=$R_A$] (3,3)
	to[short] (4.5,3)
	to[resistor=$R_C$] (7.5,3)
	to[short] (7.5,0)
	to[short] (3,0)
	to[short] (0,0)
	(3,1.5) to[short] (3,4.5)
	to[switch=$S_1$](4.5,4.5)
	to[resistor=$R_B$] (7.5,4.5)
	to[short] (7.5,1.5)
	to[resistor,l_=$R_D$] (4.5,1.5)
	to[switch,invert,l_=$S_2$] (3,1.5)

;\end{circuitikz}\end{center}
\end{figure}


\begin{pycode}
from pylab import *
import random

from randassign import RandAssign
ra = RandAssign()

def make_problem():
	# Random values of circuit components
	def rand_battery():
		return random.choice([1.5, 3, 4.5, 6, 9, 12, 15, 18])
	def rand_R1():
		return random.choice([100, 120, 150, 180, 220])
	def rand_R2():
		return random.choice([220, 270, 330, 390, 470, 560, 680, 820, 1000])
	def rand_R3():
		return random.choice([470, 560, 680, 820, 1000])	
	def rand_choice1():
		return random.choice(["current through","potential difference across"])
	def rand_choice2():
		return random.choice(["$R_A$","$R_C$"])
	def rand_choice3():
		return random.choice(["greatest","smallest"])

		
	RA = rand_R1()
	RB = rand_R2()
	RC = rand_R2()
	RD = rand_R2()
	V1 = rand_battery()
	choice1 = rand_choice1()
	choice2 = rand_choice2()	
	choice3 = rand_choice3()
	
	problem_template = r'''
    \textbf{{Given the following values}}:
    $V=\SI{{{4:.3g}}}{{\volt}}$,
    $R_A=\SI{{{0:.4g}}}{{\ohm}}$,
    $R_B=\SI{{{1:.4g}}}{{\ohm}}$,
    $R_C=\SI{{{2:.4g}}}{{\ohm}}$, and
    $R_D=\SI{{{3:.4g}}}{{\ohm}}$, determine how the switches should be set so that the {{{5}}} {{{6}}} is {{{7}}}.
	'''
	
	problem = problem_template.format(RA,RB,RC,RD,V1,choice1,choice2,choice3)
	
	return(problem)
	
## Generate 4 unique problems
## The chance of two identical problems is very low in this case
## But checking for identical problems won't hurt
#problems = []
#while len(problems) < 4:
#    p = make_problem()
#    if p not in problems:
#        problems.append(p)
#        t = '\\SI{{{0:.3g}}}{{\\micro\\farad}} and \\SI{{{1:.3g}}}{{\\micro\\coulomb}}and \\SI{{{2:.3g}}}{{\\micro\\coulomb}} and \\SI{{{3:.3g}}}{{\\micro\\coulomb}} and \\SI{{{4:.3g}}}{{\\volt}} and \\SI{{{5:.3g}}}{{\\volt}} and \\SI{{{6:.3g}}}{{\\volt}} and \\SI{{{7:.3g}}}{{\\joule}} and \\SI{{{8:.3g}}}{{\\joule}} and \\SI{{{9:.3g}}}{{\\joule}}'
#        ra.addsoln(t.format(c123, q1, q2, q3, v1, v2, v3, u1, u2, u3))
#
#print('\\begin{enumerate}')
#for n, p in enumerate(problems):
#    print('\\item ' + p)
#    if n+1 < len(problems):
#        print(r'\vspace{1in}')
#print('\\end{enumerate}')

p = make_problem()
#t = '\\SI{{{0:.3g}}}{{\\micro\\farad}} and \\SI{{{1:.3g}}}{{\\micro\\coulomb}}and \\SI{{{2:.3g}}}{{\\micro\\coulomb}} and \\SI{{{3:.3g}}}{{\\micro\\coulomb}} and \\SI{{{4:.3g}}}{{\\volt}} and \\SI{{{5:.3g}}}{{\\volt}} and \\SI{{{6:.3g}}}{{\\volt}} and \\SI{{{7:.3g}}}{{\\joule}} and \\SI{{{8:.3g}}}{{\\joule}} and \\SI{{{9:.3g}}}{{\\joule}}'
#ra.addsoln(t.format(c123, q1, q2, q3, v1, v2, v3, u1, u2, u3))

print(p)

\end{pycode}
\question Explain whether or not your answer to the previous question would change if all the resistors were identical.
\end{questions}
	
\end{document}
